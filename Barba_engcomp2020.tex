\documentclass[10pt,journal,compsoc]{IEEEtran}

\usepackage{amsmath}
\usepackage[nocompress]{cite}
\usepackage{color}
\usepackage[caption=false,font=normalsize,labelfont=sf,textfont=sf]{subfig}
\usepackage{stfloats}
\usepackage{url}
\usepackage{multirow}

\definecolor{lightgray}{rgb}{0.95,0.95,0.95}

\usepackage[pdftex]{graphicx}
\graphicspath{{figures/}}
\DeclareGraphicsExtensions{.pdf,.jpeg,.png}

% correct bad hyphenation here
\hyphenation{op-tical net-works semi-conduc-tor}

\begin{document}

% title
\title{Engineers Code: re-usable, open educational modules for engineering undergraduates}
% author names and affiliations
\author{Lorena A. Barba
\IEEEcompsocitemizethanks{\IEEEcompsocthanksitem Mechanical and Aerospace Engineering,
the George Washington University, Washington, DC 20052.\protect\\
% note need leading \protect in front of \\ to get a newline within \thanks as
% \\ is fragile and will error, could use \hfil\break instead.
Email: labarba@gwu.edu}% <-this % stops an unwanted space
%\thanks{Manuscript submitted 2019}
}

\IEEEtitleabstractindextext{%
\begin{abstract}

\end{abstract}
}

% make the title area
\maketitle

\IEEEraisesectionheading{\section{Introduction}\label{sec:introduction}}

\IEEEPARstart{S}{cience} and engineering undergraduate programs routinely include in their curriculum a basic programming course, often provided as a service course by the local department of computer science. 
Regardless of the programming language used, one common approach for teaching non-CS (computer science) majors is to simplify the standard introductory CS course, to create a ``lightweight'' version focusing on programming basics. 
Less common---even though it is known to be effective---is to teach programming \emph{in context}. 
The classic example is the media-computing course introduced nearly two decades ago at the Georgia Institute of Technology (for liberal arts, architecture, and management/business majors) \cite{guzdial2003media,guzdial2005design}. 
Evaluation efforts on that multi-year curricular innovation support the idea that context-based teaching of programming increases student motivation and success \cite{forte2005motivation,guzdial2013exploring}. 
(It also reduced the success gender gap.) 
A more recent effort to introduce contextualized computing education in engineering found that it was effective in enabling students to apply computational practices to continue learning in their discipline \cite{magana2016case}. 
In view of their observations, the authors recommended integrating context-based computing early and often in the engineering curriculum. 
In this paper, I describe an initiative to develop a series of learning modules meant to integrate computing in the undergraduate engineering curriculum. 
The modules adopt the context-based format for teaching programming, and are also designed to be re-usable, and shared under standard public licenses (CC-BY for content and BSD-3 for code). 
We first developed three learning modules---each adding up to about one university credit of work---and taught a second-year engineering course based on this content  in Fall 2017 and Fall 2018. 
The faculty of the Mechanical and Aerospace Engineering department then approved to create a two-course series in computing, and we used the first two modules for a first-year course in Spring 2019, and started writing two additional learning modules to complete a revised second-year course. 
We are working with colleagues to develop additional learning modules to use within core engineering courses, aiming to reinforce the ability of students to use computational practices in their discipline. 



\bigskip

\textbf{Lorena A. Barba} is a professor of mechanical and aerospace engineering at the George Washington University. Her research interests include computational fluid dynamics, biophysics, and high-performance computing. She is co-Editor of the CiSE Reproducible Research Track, Associate Editor for The ReScience Journal, Associate Editor-in-Chief of the Journal of Open Source Software, and Editor-in-Chief of the Journal of Open Source Education. Barba received a PhD in aeronautics from the California Institute of Technology. Contact her at labarba@gwu.edu.

\bibliographystyle{IEEEtran}
% argument is your BibTeX string definitions and bibliography database(s)
\bibliography{comp_edu}
%
% <OR> manually copy in the resultant .bbl file
%\begin{thebibliography}{1}
%\bibitem
%\end{thebibliography}

\end{document}